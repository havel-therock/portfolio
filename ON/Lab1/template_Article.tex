\documentclass[]{article}
\usepackage[utf8]{inputenc}
\usepackage[T1]{fontenc}
\usepackage{listings}
\usepackage{xcolor}

%opening
\title{Obliczenia naukowe. Lista nr 1. Sprawozdanie.}
\author{Kacper Szatan nr 236478}

%%
%% Julia definition
%%
\lstdefinelanguage{Julia}%
{morekeywords={abstract,break,case,catch,const,continue,do,else,elseif,%
end,export,false,for,function,immutable,import,importall,if,in,%
macro,module,otherwise,quote,return,switch,true,try,type,typealias,%
using,while},%
sensitive=true,%
alsoother={\$},
morecomment=[l]\#,%
morecomment=[n]{\#=}{=\#},%
morestring=[s]{"}{"},%
morestring=[m]{'}{'},%
}[keywords,comments,strings]%
\lstset{%
	language         = Julia,
	basicstyle       = \ttfamily,
	keywordstyle     = \bfseries\color{blue},
	stringstyle      = \color{magenta},
	commentstyle     = \color{gray},
	showstringspaces = false,
}

\begin{document}
\maketitle
\section{CEL}
Celem listy jest zapoznanie się z językiem programowania \textit{JULIA}, oraz z reprezentacją liczb zmiennoprzecinkowych w standardzie \textbf{IEEE 754}. Należało wykonać 7 zadań, które miały skonfrontować naszą wiedzę zdobytą na wykładzie dotyczącą reprezentacji liczb oraz sposobu wykonywania działań w komputerze.
\section{REALIZACJA}
Poniżej zajmiemy się omówieniem realizacji zadań wykonanych na laboratorium. Wnioski wyciągniętę z przeprowadzonych eksperymentów będą podane w następnym punkcie (\textbf{3. WNIOSKI}).   
\subsection{Zadanie 1 (Rozpoznanie arytmetyki)}
Całe zadanie testowano w standardach zgodnych z \textbf{IEEE 754} (half, single, double). W pierwszej części zadania musimy iteracyjnie wyznaczyć \textit{macheps} \footnote{Epsilonem maszynowym macheps (ang. machine epsilon) nazywamy najmniejszą liczbę macheps > 0 taką, że fl(1.0 + macheps) > 1.0.}. Następnie porównać otrzymane wyniki z wbudowaną funkcją \textit{eps(typ arytmetyki)} oraz z danymi w pliku nagłówkowym \textit{float.h} z dowolnej instalacji języka C.

\begin{lstlisting}
x = Float16(1)
y = Float32(1)
z = Float64(1)
\end{lstlisting}
\textit{Listing 1.1: Zmienne dla całego zadania pierwszego.}
\newpage
\begin{lstlisting}
function machEpsEx(one)
mach_eps = one	
#przypisujemy mach_eps jedynke 
#w arytmetyce ktora testujemy
checker = mach_eps
while(one + checker > one)
	mach_eps = checker
	checker = checker / 2
	typeof(one)(checker)
	typeof(one)(mach_eps)
end
#return mach_eps #gdyby byl potrzebny macheps to odkomentowac.
# ponizej ladny format drukujacy
str = string("Calculated macheps ", typeof(mach_eps), " = ",
mach_eps, "\neps(", (typeof(one)), ") = ", eps(typeof(one)), "\n")
print(str)
end
\end{lstlisting}
\textit{Listing 1.2: Funkcja pozwalająca obliczyć macheps iteracyjnie.}
\\\\
W drugiej części zadania mamy iteracyjnie wyznaczyć liczbę \textit{eta} \footnote{eta to liczba, taka że eta > 0.0}. Następnie porównać otrzymane wyniki dla wywołania funkcji wbudowanej \textit{nextfloat(typ reprezentacji(0.0))}. 
\begin{lstlisting}
function etaEx(one)
zero = typeof(one)(0.0)
eta = one
checker = eta
while(checker > zero)
	eta = checker
	checker = checker / 2
	typeof(one)(checker)
	typeof(one)(eta)
end
#return eta #gdyby byla potrzebna eta to odkomentowac.
# ponizej ladny format drukujacy
str = string("Calculated eta ", typeof(eta), " = ",
eta, "\nnextFloat(", (typeof(one)), "(0.0)) = ", nextfloat(typeof(zero)(zero)), "\n",
"floatmin(", (typeof(one)), ") = ", floatmin(typeof(zero)), "\n")
print(str)
end
\end{lstlisting}
\textit{Listing 1.3: Funkcja pozwalająca obliczyć eps iteracyjnie. (Analogiczna do funkcji na Listingu 1.2)}
\newpage 
Trzecia część zadania polega na iteracyjnym wyznaczeniu MAX. Skorzystano tu z wbudowanej funkcji \textit{isinf(liczba)}, sprawdzającej czy podana liczba jest nieskończonością. Na początku obliczamy maksymalną liczbę, która jest postaci \(x = 2^k \: k\epsilon \mathcal{Z}\). Następnie aby obliczyć właściwe maksimum szukamy liczby postaci \(MAX = x * (2 - m)\). Gdzie \(m\) jest najmniejszą liczbą przy której zaprezentowane wyrażenie nie jest nieskończonością. Otrzymane wyniki należało porównać wywołaniem funkcji systemowej \textit{floatmax(typ reprezentacji)} oraz z danymi zawartymi w pliku nagłówkowym \textit{float.h} dowolnej instalacji języka C. 
\begin{lstlisting}
function maxEx(one)
max = one
checker = max
while(!isinf(checker))
	max = checker
	checker = checker * 2
	typeof(one)(checker)
	typeof(one)(max)
end
min = etaEx(one)
while(isinf(max * (2 - min)))
	min = min * 2    
end
max = max * (2 - min)
#return max #gdyby byla potrzebna eta to odkomentowac.
# ponizej ladny format drukujacy
str = string("Calculated MAX ", typeof(max), " = ",
max, "\nfloatmax(", (typeof(one)), ") = ", floatmax(typeof(one)), "\n")
print(str)
end
\end{lstlisting}
\textit{Listing 1.4: Funkcja pozwalająca obliczyć MAX iteracyjnie.}
\subsection{Zadanie 2}
Kahan stwierdził, że epsilon maszynowy (macheps) można otrzymać obliczając wyrażenie
\(3(4/3-1)-1\) w arytmetyce zmiennopozycyjnej. W tym zadaniu eksperymentalnie sprawdzamy poprawność tego twierdzenia, dla wszystkich typów reprezentacji (half, float, double).
\newpage
\begin{lstlisting}
function machEpsKahan(one)
# Kahan formula 3(4/3-1)-1 
tmp1 = typeof(one)(typeof(one)(4) / typeof(one)(3)) # 4 / 3 = tmp1
tmp2 = typeof(one)(tmp1 - typeof(one)(1)) # tmp1 - 1 = tmp2
tmp3 = typeof(one)(typeof(one)(3) * tmp2) # 3 * tmp2 = tmp3
mach_eps = typeof(one)(tmp3 - typeof(one)(1)) # tmp3 - 1 = mach_eps
return mach_eps
end
\end{lstlisting}
\textit{Listing 2.1: Funkcja sprawdzająca poprawność twierdzenia Kahan'a.}
\subsection{Zadanie 3}
W zadaniu trzecim mamy do zbadania rozmieszczenie liczb w artmetyce Float64 w zadanych przedziałach [0.5 , 1], [1 , 2], [2 , 4]. Wywołanie napisanej funkcji z odpowiednimi parametrami pozwala nam zbadać co się dzieje przy przejściu z przedziału do przedziału i jaki jest aktuala odległość między liczbami. Użyto bitstring do porównania delt.
\begin{lstlisting}
x = Float64(0.999999)
y = Float64(1.00001)
x2 = Float64(1.999999)
y2 = Float64(2.00001)
x3 = Float64(3.999999)
y3 = Float64(4.00001)
function distribution(x, y)
delta = Float64(nextfloat(x) - x)
print("Start delta = ", delta, "\n")
while(x < y)
	x = Float64(x + delta)
	if(x + delta == x)            
	 print("For x = ", x, " delta = ", delta, " is to low.\n",
	 "Distribution of floats is now wider.\n",
	 "new delta = old delta * 2 === ", delta * 2, "\n")
	 print("BITSTRING delty = ", delta, "\n", bitstring(delta), "\n")
	 delta = delta * 2
	 print("BITSTRING delty = ", delta, "\n", bitstring(delta), "\n")
		break
	end
end
end

distribution(x, y) 
distribution(x2, y2) 
distribution(x3, y3) 
\end{lstlisting}
\textit{Listing 3.1: Funkcja sprawdzająca rozmieszczenie liczb w zadanych przedziałach.}
\subsection{Zadanie 4}
W zadaniu czwartym należało eksperymentalnie w arytmetyce Float64 wyznaczyć liczbę zmiennopozycyjną \(x\) w przedziale \(1 < x < 2\), taką, że \(x*(1/x) \not = 1\). Znaleziona liczba powinna być możliwie jak najmniejsza.
\begin{lstlisting}
x = Float64(1)
function find(x)
while (x < Float64(2))
	if (Float64(x * (Float64(1 / x))) != Float64(1))
	println(x)
	break
	end
x = nextfloat(x)
end
end
\end{lstlisting}
\subsection{Zadanie 5}
W zadaniu 5 należy zaimplementować cztery algorytmy dla podanych danych:
\[x = [2.718281828, -3.141592654, 1.414213562, 0.5772156649, 0.3010299957]\]
\[y = [1486.2497, 878366.9879, -22.37492, 4773714.647, 0.000185049]\]
\begin{itemize}
	\item "w przód" \( \sum_{i = 1}^{n = 5}  x_iy_i \)
	\begin{lstlisting}
	function sumA(x, y, n)
	s = typeof(x[1])(0)
	for i = 1: n
		s = s + typeof(x[1])(x[i] * y[i])
		s = typeof(x[1])(s)
	end
	return s  
	end
	\end{lstlisting}
	\textit{Listing 5.1: funkcja obliczająca sumę w przód}
	\item "w tył" \( \sum_{i = n = 5}^{1}  x_iy_i \) Kod sumy w tył nie został zamieszczony gdyż jest analogiczny i prawie identyczny do sumy w przód.
	\item od największego do najmniejszego (dodaj dodatnie liczby w porządku od największego do najmniejszego, dodaj ujemne liczby w porządku od najmniejszego do największego, a następnie daj do siebie obliczone sumy częściowe).
	\item  od najmniejszego do największego (przeciwnie do metody (c)).
\begin{lstlisting}
function sumD(x, y, n)
t = sort(newTab(x, y, n))
sum_positive = typeof(x[1])(0)
sum_negative = typeof(x[1])(0)
for i = 1 : n
	if i > 2 # DWIE PIWERWSZE LICZBY SA UJEMNE
		for j = i - 1 :-1: 1
		 sum_negative = typeof(x[1])(sum_negative + t[j])
		end
		for k = i : n
		 sum_positive = typeof(x[1])(sum_positive + t[k])
		end
		break
	end
end
return typeof(x[1])(sum_negative + sum_positive)
end
\end{lstlisting}
\textit{Listing 5.2: Funkcja pozwalająca obliczyć sumę w zaczynając od najbliższych zera wartości. Obliczono sumy częściowe ujemną i dodatnią odzdzielnie i na końcu je dodano.}
\end{itemize}
Wyniki należało porównać z prawidłową wartością (dokładność do 15 cyfr) \(-1.00657107000000 * 10^{-11}\).

\subsection{Zadanie 6}
Zadanie szóste Policz polega na policzeniu w arytmetyce Float64 wartości następujących funckji \( f(x) = \sqrt{x^2 + 1} - 1\quad\)\(g(x) = \)\(x^2\over{\sqrt{x^2 + 1} + 1}\), dla \( x = 8^{-1}, 8^{-2}, 8^{-3}, ... \). Chociaż \(f = g\) komputer daje różne wyniki. Trzeba sprawdzić, które z nich są wiarygodne, a które nie.
\begin{lstlisting}
function f(x)
 return Float64(Float64(sqrt(Float64(square(Float64(x)))+1.0))-1.0)
end
function g(x)
 return Float64(Float64(square(Float64(x))) /
 Float64((Float64(sqrt(Float64(square(Float64(x)))+1.0))+1.0)))
end
function z6()
for i=1:20
	println("f(8^(-", i, ")) = ", f(8.0^(-i)))
	println("g(8^(-", i, ")) = ", g(8.0^(-i)))
end
end
\end{lstlisting}
\textit{Listing 6.1: Funkcja pozwalająca obliczyć wynik f(x) i g(x) i porównać wyniki.}
\subsection{Zadanie 7}
W ostatnim zadaniu należało obliczyć przybliżoną wartość pochodnej \(f(x)\) w punkcie \(x\) ze wzoru: \[ f^`_0(x_0) \approx  \tilde{f}^`_0(x_0) = \frac{f(x_0 + h) + - f(x_0)}{ h}\]
Dla funkcji \(f(x) = sin(x) + cos(3x)\) w punkcie \(x_0 = 1\). \newline Obliczyć także błąd \(|f^`_0(x_0) - \tilde{f}^`_0(x_0)|\). \newline
Przeprowadzić badanie dla \(h = 2^{-n} \: (n = 0,1,2,3,...54)\)
\begin{lstlisting}
function f(x)
	return Float64(sin(x) + cos(3.0 * x))
end
function derivativeReal(x) 
	return Float64(cos(x) - 3.0 * sin(3.0 * x))
end
function derivativeAprox(x,h) 
	return Float64((Float64(f(Float64(x+h)) - f(x))) / Float64(h))
end
function error()
for n = 0:54
 h = Float64(2^(Float64(-n)))
 println("Error for h=2^(-", n, ") = ",
 abs(derivativeReal(1) - derivativeAprox(1, h)))
end
end
function printDeriv()
	for n = 0:54
	 h = Float64(2^(Float64(-n)))
	 println("Derivative approximation for h=2^(-", n, ") = ",
	 derivativeAprox(1, h))
	end
end
\end{lstlisting}
\textit{Listing 7.1: Zestaw funkcji pozwalających na wyliczenie przybliżenia pochodnej w punkcie jak i jej dokładnej wartości. Funkcja error() potrafi podać błąd jaki został popełniony podczas przybliżenia.}
\section{WYNIKI I WNIOSKI}
\subsection{Zadanie 1}
\begin{table}[h!]
	\centering
	\begin{tabular}{||c c c c c c c||} 
		\hline
		Precyzja & macheps & eps() & float.h (macheps) & eta & nextfloat(0.0) & floatmin()\\ [0.5ex] 
		\hline\hline
		Float16 & 0.000977 & 0.000977 & - & 6.0e-8 & 6.0e-8 & 6.104e-5\\ 
		Float32 & 1.192093e-7 & 1.192093e-7 & 1.192093e-7F & 1.0e-45 & 1.0e-45 & 1.175e-38\\
		Float64 & 2.220446e-16 & 2.220446e-16 & 2.220446e-16L & 5.0e-324 & 5.0e-324 & 2.225e-308\\
		\hline
	\end{tabular}
\caption{Wyniki z zadania pierwszego. Widać, że eta odpowiada \(MIN_{SUB}\), a floatmin() \(MIN_{NOR}\).}
\end{table}
Z tabelki 1 możemy odczytać, że wraz ze wzrostem precyzji \textit{macheps} maleje. Precyzja arytmetyki którą określaliśmy epsilonem to właściwie macheps podzielony przez dwa.
\(MIN_{SUB}\) to faktyczne najmniejsza liczba jaką możemy zapisać w komputerze.
Jest ono postaci \(2^{-(t-1)}2^{C_{min}}\). 
\(MIN_{NOR}\) to minimum znormalizowane postaci \(2^{C_{min}}\).
\subsection{Zadanie 2}
\begin{table}[h!]
	\centering
	\begin{tabular}{||c c c||} 
		\hline
		Precyzja & macheps & Kahan \\ [0.5ex] 
		\hline\hline
		Float16 & 0.000977 &  -0.000977\\ 
		Float32 & 1.192093e-7 & 1.192093e-7\\
		Float64 & 2.220446e-16 & -2.220446e-16 \\
		\hline
	\end{tabular}
	\caption{Wyniki wyliczeń macheps metodą Kahan'a.}
\end{table}
Wyniki są równe co do wartości bezwzględnej. W precyzji Float16 i Float64 otrzymujemy wynik o przeciwnym znaku.
\subsection{Zadanie 3}
\subsection{Zadanie 4}
\subsection{Zadanie 5}
\subsection{Zadanie 6}
\subsection{Zadanie 7}
Jak wytłumaczyć, że od pewnego momentu zmniejszanie wartości h nie poprawia przybliżenia wartości pochodnej? Jak zachowują się wartości 1+h? Obliczone przybliżenia pochodnej
porównać z dokładną wartością pochodnej, tj. zwróć uwagę na błędy |f

\end{document}
